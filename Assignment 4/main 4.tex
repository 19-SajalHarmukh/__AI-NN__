\documentclass{article}
\usepackage[utf8]{inputenc}
\PassOptionsToPackage{hyphens}{url}\usepackage{hyperref}
\setlength{\parindent}{4em}
\setlength{\parskip}{1em}

\title{Quashing the Bermuda Triangle theory}
\author{Sajal Harmukh}
\date{6 August 2021}

\begin{document}

\maketitle

\section*{Introduction}
The Bermuda triangle is a region of the Atlantic Ocean that lies between Bermuda, Puerto Rico and (in its most popular version) Florida. Ship and aircraft disasters and disappearances perceived as frequent in this area have led to the circulation of stories of unusual natural phenomena, paranormal encounters and interactions with extraterrestrials, referred to as the interactions with the hypothetical life which might not have originated from the earth.
\newline
The earlier suggestion of the disappearance in the Bermuda region appeared in 17, 1950, article published in the The Miami Herald by Edward Van Winkle Jones. Just two years later Fate magazine published an article named "Sea mystery at our backdoor" by George Sand covering the loss of several ships which included the loss of flight 19 (a group of five US NAVY torpedo bombers on a training mission)
\newline

\section*{Use of Artificial Intelligence to Bust the Myth}
Since, there were numerous assumptions made to justify the scenario of lost ships and flights
The use of Artificial intelligence could come handy in tackling this scene.Although the approach might seem abstract and skeptical, but may lead to conclusive results:
\begin{itemize}
\item Designing a Processor/system that is based on natural language processing and which could be fed by the data of the lost flights and ships over the region of bermuda triangle and the data of the lost ships and flights over the other parts of the world .
\item The comparitive data analysis could be done on the input data set and abnormalities could be found between different technical parameters .
\item Use the precision and accuracy of the algorithm such that the assumptions made over the year could be backed up by some concrete proof . weather forecast abnormalities over this region has always been a matter of interest.
\item Once the algorithm is made standard according to incidents that have happened over the period of time .The under lying pattern or the functioning of lost ships or flights in Bermuda triangle and sme other parts.
\item Then Design a simulation using the data provided by the algorithm to compare is there any such difference observed over bermuda triangle area.

\end{itemize}

\section*{References:}
\url{https://en.wikipedia.org/wiki/Bermuda_Triangle}


\end{document}