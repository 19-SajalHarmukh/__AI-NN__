\documentclass{article}
	\usepackage[utf8]{inputenc}
	
	\title{\vspace{-2cm}Moravec's paradox Summary}
	\author{Sajal Harmukh }
	\date{28 July 2021}
	
	\begin{document}
	\maketitle 
	\paragraph{ he Moravec's paradox is an idea articulated by Hans Moravec among many others, which states that it is easier to design an artificial system which can perform high-level intelligent tasks, than to make the system perform tasks which generally don't involve cognitive skills when performed by humans. Moravec's paradox is the observation that, contrary to popular belief, thinking takes very little processing, but sensory abilities necessitate a massive amount of computation. In the 1980s, Hans Moravec, Rodney Brooks, Marvin Minsky, and others defined the idea. "It's very easy to have computers function at adult levels on IQ tests or play checkers, but it's difficult or impossible to give them the perceptual abilities of a one-year-old."}
	
	\paragraph{ All human abilities are executed organically, with natural selection-designed equipment. Natural selection has had more time to enhance the design of an older talent. Moravec In the future, we should not anticipate it to be very effective in its execution.The lightest layer of human cognition is the conscious process we call reasoning.}
	
	\paragraph{Because the first human talents are mostly unconscious, they appear to us to be simple. According to Dr. Jodie Gorman, we can expect abilities that look straightforward to be difficult to reverse-engineer, while skills that demand effort may not be easy to reverse-engineer at all.Identifying a face, moving around in space, evaluating people's motives, catching a ball, recognising a voice, creating suitable objectives, and paying attention to fascinating things are just a few of the talents that have been evolving for millions of years. These are abilities and methods that have only been perfected over a few thousand years.}
	
	\paragraph{ It describes the future where machines will take jobs which require specialistic skills, making people serving an army of robotic chiefs and analysts. Others argue that paradox guarantees that AI will always need an assistance of people. Or, perhaps more correctly, people will use AI to improve those skills which aren’t as highly developed by nature.For sure Moravec’s paradox proves one thing — the fact that we developed computer to beat human in Go or Chess doesn’t mean that General Artificial Intelligence is just around the corner. Yes, we are one step closer. But as long as AGI means for us “full copy of human intelligence”, over time it will be only harder..}
	\end{document}